\clearpage
\pagenumbering{roman} 
    \setcounter{page}{2}
\thispagestyle{plain}
\section*{\centering RESUMEN}
\justify % Texto justificado
El presente trabajo explora estrategias para fomentar una cultura DevOps y mejorar la implementación de prácticas de Integración Continua (CI) y Entrega Continua (CD) en equipos de desarrollo de software. DevOps se define como un enfoque cultural y metodológico que busca integrar y optimizar el flujo de trabajo entre los equipos de desarrollo y operaciones, promoviendo la colaboración, comunicación y automatización. \newline

En el capítulo de antecedentes y contextualización, se destaca la evolución histórica de DevOps y su relevancia actual en la industria del desarrollo de software. Se analiza cómo DevOps aborda los desafíos de colaboración entre equipos y mejora la eficiencia en el ciclo de desarrollo, haciendo un paralelo con la revolución manufacturera de la década de 1980 y la adopción de principios Lean.\newline

En el marco teórico, se explican los fundamentos y principios de DevOps, incluyendo las tres vías fundamentales: pensamiento de flujo/sistemas, amplificación de los bucles de retroalimentación y una cultura de experimentación y aprendizaje continuo. Además, se describen las herramientas y tecnologías clave para CI/CD, como Jenkins, GitLab CI/CD y Travis CI, que facilitan la automatización de los pipelines de desarrollo y despliegue.\newline

Finalmente, se presenta una propuesta de solución que incluye estrategias para fomentar una cultura DevOps efectiva, enfocándose en liderazgo, formación, cambio cultural, selección de herramientas adecuadas, automatización y mejora continua. Se concluye que una implementación exitosa de CI/CD en un entorno DevOps puede proporcionar ventajas significativas en términos de eficiencia, calidad y competitividad en el mercado actual.\newline
\clearpage
    